
\subsection{Ausgangslage}\label{subsec:Ausgangslage}

Vor nicht allzulanger Zeit war der Bau von Prototypen oder Modellen eine Herausforderung, da es aufwändig war komplexere Formen zu modellieren und aufzubauen. 1981 wurde erstmals ein 3D-Drucker entwickelt, welcher es ermöglichte mittels Stereolithographie\footnote{\textbf{Stereolithografie} ist das älteste additive Fertigungsverfahren, bei dem ein Werkstück durch frei im Raum aufgetragene Rasterpunkte schichtenweise aufgebaut wird. Die Fertigung eines Werkstückes erfolgt vollautomatisch aus am Computer erstellten CAD-Daten.} eine komplexe Figur zu modellieren. Dies war der Anfang eines neuen technischen Zeitalters. Das Problem war jedoch, dass 3D-Drucker zu Beginn enorm teuer waren. Heutzutage sind diverse 3D-Drucker auf dem Markt, welche mittels unterschiedlicher Methoden dreidimensionale Objeke erschaffen. Diese sind mittlerweile auch für jedermann erschwinglich. Im Projekt 4 soll die Elektronik für einen 3D-Drucker entwickelt werden, welcher mittels \textit{Fused Deposition Modeling} Werkstücke kreiert. Dabei wird ein Kunststoffstrang geschmolzen und durch eine feine Düse gepresst. Der Extruder (Druckkopf) baut dann mit dem austretenden Kunststofffaden Schicht um Schicht ein dreidimensionales Objekt auf \cite{3D_Druckverfahren}\cite{Fused_Deposition_Modeling}.








