
\subsection{Ausgangslage}\label{subsec:Ausgangslage}

Die Bluetooth Technik wurde im Jahr 1998 von der "Bluetooth Special Interest Group" (SIG) als Industriestandart für Datenübertragung herausgebracht. Ursprünglich wurde das Funkverfahren jedoch von Jaap Hartsen und Sven Mattisson für die Firma Ericsson entwickelt. Der Hauptzweck dieser Methode zur Datenübermittlung war das Ersetzen von Kabelverbindungen von Mobiltelefone, Peripheriegeräte oder Computer. Der Name Bluetooth oder auf Deutsch Blauzahn kommt vom dänischen König Harald Blauzahn. Diesem König gelang es die verfeindeten Länder Dänemark und Norwegen dank seiner Kommunikationsfreudigkeit zu vereinen. Da die skandinavischen Firmen Nokia und Ericsson viel Aufwand in die Bluetooth Technologie gesteckt haben, wurde dieser Name sowie die Runen H (Harald) und B (Blauzahn) für das Logo übernommen.\cite{michna_entwicklungsgeschichte_2019} Seit dem Start von Bluetooth gab es eine Vielzahl von Versionen, die von mehreren Firmen ständig weiterentwickelt werden. Im Dezember 2009 wurde von der SIG die Version 4.0 Smart vorgestellt. Mit dieser Version von Bluetooth war es möglich kleine und sparsame Geräte wie z.B. smarte Uhren, Brillen oder sogar Ringe herzustellen.\cite{bluetooth_sig_our_2019} Ab dem Update Bluetooth 5.0 im Jahre 2016, ist es möglich Bluetooth Komponenten in einem Mesh-Netzwerk zu konfigurieren. Dieses Netzwerk basiert auf einem "many-to-many pairing system" d.h. jeder Teilnehmer ist mit den anderen Teilnehmern verbunden. Dieses dezentralisierte System hat den Vorteil, dass es kein Master Element benötigt. Fällt ein Teilnehmer aus besteht das Netzwerk trotzdem.\cite{woolley_intro_2017} Genau hier soll das Projekt 5 ansetzten. Da die Programmierung eines Mesh-Netzwerkes sehr kompliziert ist, wird dafür eine "'Open Source Software"' geschrieben, die es ermöglicht ein Netzwerk vereinfacht aufzubauen und zu konfigurieren.








