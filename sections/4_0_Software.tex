\clearpage
\section{Software}\label{sec:Software}
Das Kapitel Software befasst sich mit der Programmierung des Mikrocontroller \textit{nRF52840} sowie des \textit{Gateway Interface System}.

\subsection{Software Development Kit for nRF52}\label{subsec:SDK}
Die Software auf dem Bluetooth-Mesh-Node (siehe Kapitel \ref{subsec:BMN})  wird mit Hilfe der beiden Software Development Kits "'nRF5 SDK"'\cite{nordic_semiconductor_nrf5_2019} und "'nRF5 SDK for Mesh"'\cite{nordic_semiconductor_nrf5_2019-1} von Nordic Semiconductor entwickelt. Diese Kits enthalten eine sehr gut dokumentierte Bibliothek, die für den Quellcode des Nodes benötigt werden. Die Kits beinhalten weitere Beispiele mit Hilfe von denen die Firmware realisiert werden kann.


\subsection{Embedded Linux für Gateway}\label{subsec:LinuxfürGateway}
Mit dem Ziel den Gateway Open Source und Open Hardware zu realisieren, soll eine einfache Linux-Distribution eingesetzt werden. Beim Raspberry-Pi 4 \ref{img:raspberryPi4} eignet sich dafür besonders das Debian basierende Betriebssystem Raspbian da es eigens für die Raspberry-Pi Familie entwickelt wurde.

Basierend auf dieser Oberfläche können nun Anbindungen an Fremdsysteme wie Apple Homekit oder KNX mit den passenden Software Bausteinen realisiert werden. Welche Bausteine dies sein werden ist Teil der Umsetzung für eine spezifische Anwendung welche noch nicht festgelegt ist. Daher wird in erster Linie mit Node.js eine Ein- und Ausgabe von Daten umgesetzt. 

\subsection{Human Machine System (HMS)}\label{subsec:HMS_SW}
\todo[inline]{Löschen?}

\subsection{Open Source Projekt}\label{subsec:OSP}
Die gesamte Software im Projekt 5 wird als "'Open Source Software"' deklariert. Damit die Software global zur Verfügung steht, wird diese unter der "'General Public License Version 3"' (GPLv3) lizenziert. Die GPL beinhaltet ein starkes "'copyleft"', d.h. das Software-Projekt muss öffentlich und gebührenfrei zugänglich sein und der Quellcode muss jedem ausgehändigt werden, der danach fragt. Das starke "'copyleft"' bringt aber den Vorteil, dass die Lizenz vom Projekt nicht verändert werden kann. Wird die Software von jemandem weiterentwickelt muss dieser die GPL Lizenz weiterführen und kann kein "'closed source"' Projekt daraus erstellen. Ein weiterer Vorteil im Hinblick auf die Bachelorarbeit ist, dass das "'copyright"' einer "'open source"' Lizenz beim Entwickler bleibt. Das bedeutet die Grundlage der Software kann weiterentwickelt werden z.B. als eine Software die Hausautomation steuert und somit als "'closed source"' verwendbar ist.\cite{jaeger_was_2018}


