\clearpage
\section{Software}\label{sec:Software}
Das Kapitel Software befasst sich mit der Programmierung des Mikrocontroller \textit{NRF52840} sowie des Gateway Interface System.

\subsection{Software Development Kit for NRF52}\label{subsec:SDK}
Text Robin

\subsection{Human Machine System (HMS)}\label{subsec:HMS_SW}

\subsection{Open Source Projekt}\label{subsec:OSP}
Die gesamte Software im Projekt 5 wird als "'Open Source Software"' deklariert. Damit die Software global zur Verfügung steht, wird diese unter der "'General Public License Version 3"' (GPLv3) lizenziert. Die GPL beinhaltet ein starkes "'copyleft"', d.h. das Software-Projekt muss öffentlich und gebührenfrei zugänglich sein und der Quellcode muss jedem zur Verfügung stehen, der danach fragt. Das starke "'copyleft"' bringt aber den Vorteil, dass die Lizenz vom Projekt nicht verändert werden kann. Wird die Software von jemandem weiterentwickelt muss dieser die GPL Lizenz weiterführen und kann kein "'closed source"' Projekt daraus erstellen. Ein weiterer Vorteil im Hinblick auf die Bachelorarbeit ist, dass das Copyright einer "'open source"' Lizenz beim Entwickler bleibt. Das bedeutet die Grundlage der Software kann weiterentwickelt werden z.B. als eine Software die Hausautomation steuert und somit als "'closed source"' verwendbar ist. 