\clearpage
\subsection{Projektziele}\label{subsec:Projektziele}
In Tabelle \ref{tab:PlfichtundWunschziele} sind die Pflicht- und Wunschziele für dieses Projekt festgehalten.

\begin{table}[H]
\begin{tabular}{ | C{0.9cm} | p{3.3cm} | p{10cm} |}
	\hline
	\multicolumn{3}{|l|}{\textbf{Pflichtziele}}\\ \hline
\textbf{Nr.}& \textbf{Ziel}& \textbf{Beschrieb}\\ \hline

P1 & Bluetooth-Mesh-Netzwerk & Eine variable Anzahl an BLE-Nodes bauen ein Mesh-Netzwerk auf um darin Datenaustausch zu ermöglichen.\\ \hline

P2 & UPN & Der Universal-Peripheral-Node kann je nach Einsatz als Sensor oder Aktor konfiguriert und bestückt werden.\\ \hline

P3 & Low Power & Die UPN sind bezüglich Hardware und Software energiesparend konzipiert um sie autonom betreiben zu können.\\ \hline

P4 & Netzunabhängig & Durch Versorgung mittels Batterie und Energy-Harvesting können die UPN komplett netzunabhängig betrieben werden.\\ \hline

P5 & Energy-Harvesting & Für die Versorgung der UPN werden verschiedene Varianten für das Energy-Harvesting entwickelt. Das Ergebnis wird eine Variantenstudie sein.\\ \hline

P6 & Gateway & Zur Konfiguration des Bluetooth-Mesh-Netzwerks steht ein Gateway basierend auf Standard Hardware (Raspberry-Pi o.ä.) zur Verfügung.\\ \hline
    
P7 & LAN/WLAN & Für die Integration in TCP/IP basierte Systeme bietet der Gateway eine entsprechende Schnittstelle.\\ \hline
    	
P8 & HMI & Ein Human-Machine-Interface unterstützt den User bei der Konfiguration des Mesh-Netzwerks.\\ \hline

P9 & CLI & Mittels Command-Line-Interface kann das Mesh-Netzwerk verwaltet werden.\\ \hline

    	 \hline \hline   
 \multicolumn{3}{|l|}{\textbf{Wunschziele}}\\ \hline 	
    
    
W1 & UPN Konfiguration via Mesh & Einstellungen des UPN können via Mesh Netzwerk angepasst werden und somit z.B. die Peripheriekonfiguration verändert werden.\\ \hline

W2 & Firmwareupgrade via Mesh & Die Firmware der UPN wird via Mesh-Netzwerk aktualisiert. \\ \hline

W3 & Security & Das Mesh-Netzwerk ist gegen unerlaubten Zugriff und sonstigen Angriffen geschützt.\\ \hline

W4 & Dedizierte UPN Hardware & Das UPN ist als dedizierte Hardware realisiert und somit einsatzbereit. \\ \hline

W5 & Datenschnittstelle & Mittels passender Datenschnittstelle auf dem Gateway können Fremdsysteme wie Apple Homekit, Google Home oder KNX angebunden werden.\\ \hline

W6 & Webinterface & Für die Verwaltung des Mesh-Netzwerks und die Anbindung an Fremdsysteme steht ein Web Applikation zur Verfügung. \\ \hline

W7 & Dedizierte Gateway Hardware & Der Gateway ist auf einer dedizierten Hardware umgesetzt. \\ \hline

W8 & GSM/LTE & Für Feldanwendungen besitzt der Gateway ein GSM/LTE Modul. \\ \hline

W9 & Versuchsaufbau Energy-Harvesting & Erfolg versprechende Energy-Harvesting-Systeme werden in einem Versuchsaufbau auf deren Tauglichkeit weiter geprüft. \\ \hline


\end{tabular}\\
\caption{Projektziele}
\label{tab:Projektziele}
\end{table}



















